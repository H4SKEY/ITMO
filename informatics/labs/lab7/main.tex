\documentclass[twocolumn]{article}
\usepackage[utf8]{inputenc}
\usepackage[russian]{babel}
\usepackage{amsmath}
\usepackage{graphicx}
\usepackage{caption}
\usepackage{lipsum}

\begin{document}

\begin{minipage}[t]{0.48\textwidth}
Рассмотрим ракетный двигатель, закрепленный неподвижно, так называемый стендовый двигатель. При сгорании массы $\mu$ (мы считаем вместе и горючее, и окислитель) выделяется энергия $q\mu$, но только часть ее превращается в кинетическую энергию продуктов реакции $\mu \frac{V^2}{2}$. Поэтому $\frac{V^2}{2} < q$. Если ввести коэффициент полезного действия стендового двигателя $\eta_0$ — отношение полезной энергии к затраченной, — то получим 
\[ \mu \frac{V^2}{2} = \eta_0 q\mu, \] 
\[ V = \sqrt{2\eta_0 q}, \] 
то есть скорость истечения для данной химической реакции не может превысить ее предельной скорости истечения $V_0 = \sqrt{2q}$ (так как $\eta_0 < 1$). Рассмотрим несколько химических реакций и вычислим для них предельные скорости истечения.

1. Горение ацетилена: 
\[ 2C_2H_2 + 5O_2 = \] 
\[ = 4CO_2 + 2H_2O + 611 \text{ ккал}. \]

Число килокалорий в правой части уравнения реакции показывает, какая энергия выделяется при сгорании массы, равной сумме молей горючего и окислителя. В данном случае — при сгорании двух молей $C_2H_2$ и пяти молей $O_2$, то есть всего $2 \cdot 26 + 5 \cdot 32 = 212 \text{ г}$. Таким образом, теплотворная способность этого топлива 
\[ q = \frac{611 \text{ ккал}}{212 \text{ г}} = 2880 \frac{\text{ккал}}{\text{кг}} = 1,2 \cdot 10^7 \frac{\text{м}^2}{\text{с}^2}, \] 
а предельная скорость истечения 
\[ V_0 = \sqrt{2q} = 4,9 \frac{\text{км}}{\text{с}}. \]
\end{minipage}
\hfill
\begin{minipage}[t]{0.48\textwidth}
2. 
\[ 2H_2 + O_2 = 2H_2O + 115,6 \text{ ккал}. \] 
Вычисляем аналогично 
\[ q = 3220 \frac{\text{ккал}}{\text{кг}}; \, V_0 = 5,2 \frac{\text{км}}{\text{с}}. \]

3. 
\[ 2Al + \frac{3}{2}O_2 = Al_2O_3 + 394 \text{ ккал}; \] 
\[ q = 3860 \text{ ккал/кг}; \, V_0 = 5,6 \text{ км/с}. \]

Наибольшая из всех возможных предельная скорость истечения достигается в реакции горения металлического бериллия: 
\[ Be + \frac{1}{2}O_2 = BeO + 146 \text{ ккал}; \] 
\[ q = 5840 \text{ ккал/кг}; \, V_0 = 7 \text{ км/с}. \]

Практическому использованию этой реакции, однако, мешает то обстоятельство, что металлический бериллий и его соединения очень ядовиты. Рассмотрим теперь коэффициент полезного действия ракеты в целом, то есть отношение ее полезной энергии, равной $\frac{mv^2}{2}$, к затраченной. Последнюю с небольшой ошибкой можно считать равной $qM_0$, поскольку подавляющая часть начальной массы приходится на топливо. Таким образом, 
\[ \eta = \frac{mv^2}{2M_0q} = \frac{v^2}{2q} e^{-v/V}. \]

Или, подставляя вместо $q$ его выра-

\begin{center}
\includegraphics[width=0.8\linewidth]{figure2.png} 
\captionof{figure}{Рис. 2.} 
\end{center}
\end{minipage}

\end{document}
